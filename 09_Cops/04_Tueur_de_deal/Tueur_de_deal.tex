\documentclass[10pt,a5paper,oneside]{book}
\usepackage[utf8]{inputenc}
\usepackage[french]{babel}
\usepackage[T1]{fontenc}
\usepackage{amsmath}
\usepackage{amsfonts}
\usepackage{amssymb}
\usepackage{graphicx}
\author{Renaud Guezennec}
\title{Tombes}


\begin{document}

\maketitle \clearpage
\tableofcontents \clearpage
\listoffigures \clearpage


\chapter{Le scénario}


\section{Objectifs du scénario}

Le scénario poussera vos joueurs à affronter leur hiérarchie dans le but d’arrêter un duo de tueurs en série. Ils s’enfonceront dans les cartels pour gagner leur 
confiance afin qu'il donne les éléments les plus précis pour l’enquête. Un dilemne moral leur sera posé entre fermer les yeux sur les activités des victimes (ou leur famille) et le respect de la loi. 

\section{Introduction}

Un tueur en série est en activité, ses cibles sont des femmes autour des gros dealers de drogue de Los Angeles.
En réalité, ce n'est pas un tueur en série mais deux. Ils travaillent en duo.

\section{Prise de contact}

Paul Edmont est un dealer de drogue. Il souhaitait se ranger avec sa femme. 
Malheureusement des psychopathes l'ont enlevée et tuée. Dans un premier temps, il va chercher par lui même mais il va se rendre compte bien vite qu'il n'a pas les compétences. 
Il va se mettre à le recherche d’agents du COPS capables de l'aider discrètement.

Il se passe un mois entre le décés de sa femme et sa prise de contact avec vos joueurs. 
Dans cette intervalle, il a fait trois choses:
* Enquêter par lui même (sans grand résultat)
* Se mettre à l’abri en quittant la Californie
* Chercher des COPS répondant à son besoin.

Une fois ses taches effectuées, il contacte les joueurs.


\section{Méthodes de contact}

L'objectif de Paul s'est la discrétion à plusieurs échelles: 
Si le milieu sait que Paul parle aux flics, il met sa vie en danger.
Si la police débarque en nombre dans cette affaire, ils vont trop s'intéresser à ses activités et délaisser l’enquête.

Il cherche donc des agents capables de respecter ses deux conditions.

\subsection{Méthode douce}

Si vos joueurs ont des contacts/indic dans le milieu de la drogue ou même s'il a été montré devant les médias. Il sera choisi à condition qu'il ait la réputation d'être loyal avec ses 
indicateurs ou «réglo» avec ses contacts. Bref, il cherche des roseaux.

\subsection{Méthode Musclée }

Si vos joueurs sont plutôt des chênes rigides, Paul utilisera une approche plus dure. 
il choisira le chantage et trouvera des moyens de pression sur les personnages (vieux dossiers, famille, problème d'argent etc). 
Son but est de faire comprendre qu'il est sérieux.

\section{ L'appel}

Une fois son choix effectué, il enverra un téléphone prépayé à un de vos personnages. 
Celui qui correspond le mieux à son besoin. Ce personnage sera considéré comme l'enquêteur principal par Paul. 

Le COPS recevra le téléphone, par courrier, par l'intermédiaire d'un indic/contact, ou une enveloppe dans sa voiture…
Sur le téléphone, il y a le code PIN écrit sur un post-it (Cela sera le numéro du badge du COPS en question).
Dans le téléphone, il y a un seul numéro enregistré « Appelez moi ».

Si Paul a utilisé la méthode musclée, il y aura dans le téléphone des photos ou des preuves des moyens de pression qu'il a.

Quand le personnage appelle, Paul présentera des excuses pour ses manières (surtout si méthode musclée) mais restera froid. Il expliquera sa situation. 
Il cherche à recruter des enquêteurs sérieux pour trouver les meurtrier de sa femme. Paul pense avoir trouver le bon avec ce COPS. S'il a besoin il peut faire appel à des collègues de 
confiance sans mettre au courant la police et un procureur. 

Il ne donnera ni son nom, ni sa profession. Il ne souhaite pas voir des flics fouiller sa vie. Il parle d'un ton sérieux et froid. Il cache une profonde colère. 

Il énonce les faits (voir section suivante), il demandera au PJ s'il est prêt à l'aider et s'il connaît des flics de confiance. Il se renseignera sur eux (just in case).
S'il accepte, Paul répondra aux questions sur l’enquête.
Si les PJ force un peu pour obtenir des informations pratiques sur la résidence de la victime ou autre en m’étant en évidence que c'est utile pour l’enquete. Il lâchera des informations. 
Attention le RP doit être amical ou neutre.
Si le cops s’énervent au téléphone, il gueulera plus fort et raccrochera.

Paul peut proposer une somme d'argent assez confortable pour motiver un cops indécis (attention si le SAD l’apprend). 

S'il est nécessaire, Paul mettra de son côté toutes les chances et contactera un autre PJ.

Encadre: Enquêteur principal pour Paul
 

\subsection{les faits}

Paul racontera qu’Helen (sa femme) a été enlevée (Il donne le prénom de femme). Il a reçu de la part des ravisseurs plusieurs enregistrements et coups de téléphone. Le ravisseur 
souhaitait une rançon de 400 000\$.
Il a payé la rançon. Le ravisseur a joué avec lui. Il a traversé la ville pour aller dans un coin pourri de Skid raw. Il a déposé sa voiture ouverte avec l’argent puis à marcher 5 mins 
pour revenir après. Il devait retrouver sa femme dans la voiture mais elle n'était pas là. Plusieurs heures après au milieu de la nuit, il a reçu un appel, il a eu un nouveau lieu. 
Cette fois en plein South Central. Une voiture abandonnée. Sa femme était dans le coffre coupée en morceaux, emballés dans des petits sacs plastiques bleu.


\section{Chronologie}

\begin{itemize}
\item J1 - 14h00, la jeune Helen sort de chez elle, elle dit bonjour à son gardien d'immeuble et s'en va faire des courses.
\item J1 - 14h34, elle vient acheter des fleurs à son fleuriste de quartier. En sortant, du fleuriste, elle est enlevée dans un gros van blanc et bleu. C'est très rapide. 
\item J1 - 15h00, la police vient prendre les dépositions du fleuriste qui a vu la scène. Elle s'est passée devant son magasin. La police lance une enquête mais attendra d'avoir un signalement de disparition. Après tout, c'est peut-être se nouveau jeu à la mode "Kidnap me if you can". 
\item J1 - 20h00, Paul reçoit le téléphone et la première demande de rançon. 
\item J3 - 18h00, il a réuni les fonds et est prêt à payer.
\item J3 - 20h00, il est à skid raw, laisse sa voiture ouverte à un endroit. 
\item J3 - 20H10, il revient à sa voiture, pneus crevés, argent disparue et pas d’Helen.
\item J4 - 00h53, il reçoit l'appel lui disant que sa femme l’attend dans un terrain vague de North folk.  Attention aux chiens !!!
\item J4 - 02h00, il trouve la vieille voiture et dans le coffre sa femme en morceaux.
\end{itemize}


\section{Construction groupe}

Vos joueurs doivent enquêter sur cette histoire en plus de leur enquête en cours. 
Il est intéressant de découvrir comment le PJ va se débrouiller pour inviter les autres à l'aider.

\section{Début de l'enquête}

Pour débuter l’enquête, il faut comprendre ce qui s'est passé. Les PJ peuvent demander des précisions sur l'emploi du temps. Paul dira qu’Helen ne travaillait pas ce jour là. 
Ils avaient un dîner de prévu le soir. Elle a fait des courses probablement pour les invités. 

\section{Emploi du temps de la victime}

Vous pouvez imaginez un emploi du temps complet. Il peut être dur à reconstituer. Les principaux témoins pour avoir des infos sont: Paul, le concierge de l'immeuble du couple et son amie Lynn.

 
(heure de départ de l'immeuble et le concierge sait également qu'elle partait pour faire des courses car son amie Lynn devait venir avec son mari pour le diner). 
 
En enquêtant sur les habitudes, il est facile de connaitre ses magasins habituels (Paul parlera d'une rue charmante de Pasadena) et Lynn aura le nom de la rue précise. Elle affectionne une petite rue marchande de Pasadena. Il convient de faire une petite visite dans la rue. Il faut faire une enquête de voisinage.


\section{Enquête de voisinage}
\begin{itemize}
\item Le fleuriste indiquera qu'il n'a pas vu le kidnapping mais il parlera d'un van bleu avec une inscription de plomberie. Qui a fait du bruit (démarrage fort, portes qui claquent).
\item l’épicerier coréen se souvient d’Helen, elle lui a acheté pas mal d'articles. Elle est sortie pour aller au fleuriste. Il a aussi remarqué la camionnette bleu, il y avait une inscription d'une entreprise d'électricité.
\item Le caviste parlera lui aussi de la camionnette. Il a vu l'enlèvement. Deux personnes sont sorties, ont attrapé la fille. Il a appelé la police. Ils ont récolter les dépositions. Depuis, rien ne se passe. 
\end{itemize}
Helen a été enlevée dans une rue marchande de Pasadena. Elle venait d’acheter des fleurs. 
Il est possible pour les COPS de rechercher les données de sa carte banquaire mais sans mandat 
cela va être compliqué. S'il demande un mandat, on va leur demandé pour quel enquête etc...
Bref, c'est le début d’emmerdes plus gros qu'eux.

S'ils y pensent il est possible d'avoir le nom de famille d’Helen: Edmont (et donc celui de Paul).

\subsection{Enquête sur Paul}

\section{ Les autres victimes}

\subsection{ Linsey Danovan}


\subsection{Margaret Svalana}


\section{ Le mec du Zoo}

\section{L'enlevement de la fille}

\subsection{ Rencontre avec Paul}

\section{Surveillance des services}




\end{document}