\documentclass[10pt,a4paper]{book}
\usepackage[utf8]{inputenc}
\usepackage[francais]{babel}
\usepackage[T1]{fontenc}
\usepackage{amsmath}
\usepackage{amsfonts}
\usepackage{amssymb}
\usepackage{graphicx}
\author{Renaud Guezennec}
\title{Cahier de personnage}
\date{}

\newcommand\dff[3]{
\fbox{\begin{minipage}{0.4\textwidth}
\section*{Défis:}
#1
\section*{Focus:}
#2
\section*{Frappe:}
#3
\end{minipage}}
\vspace{0.5cm}
}

\begin{document}

\maketitle \clearpage
\tableofcontents \clearpage


\chapter{Introduction}

Cet ouvrage regroupe un ensemble de personnage pour le jeu «Livre des Cinq Anneaux». Les personnages sont classés par clan. Les caractéristiques des personnages sont en accords avec la quatrième édition de jeu.

\chapter{Le clan du Crabe}
\section{Bushi}

\subsection{Toritaka Toshigawua}
\subsubsection*{Ancien combattant du mur}

\paragraph{Le crabe défends l’Empire contre l’Outremonde, tel est son devoir. Toritaka Toshigawua s’est toujours préparé pour cette tâche. Il est un fier combattant du clan du crabe. Ses parents, sa famille, son sensei, tout le monde attendait de grandes choses de sa part. \\
Après son gempukku, il demanda la permission de servir sur le mur. C’était un privilège que servir sur ce mur. Il était sur d'y accomplir la destiné que son entourage lui donnait.\\
Hélas, elle a choisi un chemin plus sinueux. A peine, trois jour après son arrivée sur le mur, une attaque a été lancé par les forces de l’Outremonde. \\
Il s'est battu comme ses frères pour résister aux vagues d’assauts. \\
Seulement, un Oni s’est finalement montrer. Voyant une occasion de briller, il se jeta dans sur cet adversaire et fut défait comme un insecte insignifiant.\\ Il survécu, mais sa jambe droite n'est plus aussi mobile qu’avant. Il a donc été réformé du service sur le mur. Etre capable d’éviter un projectile ou de sauter sur un Oni de trois mètre de haut sont des conditions vitales sur le mur.\\
Le clan du crabe chercha un moyen pour Toritaka Toshigawua de faire son devoir. Le seul moyen fut pour lui d’épouser une Shinjo. La jeune femme était de bonne famille et il la trouva ravissante. Dans le couple, il assuma le role d’intendant de la maison. Son épouse assura des missions diverses et variées dans des contrées lointaine. Il s’occupe de l’éducation des enfants en attendant leur entrée dans un dojo.\\
Il reve d’action, cela lui manque. Sa femme prend beaucoup de risque et lui reste à la maison. Il cherche à utiliser ses talents de soldat.\\ Sa frustration le fera peut-être passer à l’acte de façon disproportionnée. \\
}
\dff{Deux yorikis d’emeuraude ont disparues.
}{Leur route devait les mener sur le domaine de Shinjo (Toritaka) Toshigawua.}{Shinjo Toshigawua a bien tué ces yorikis, mais il prétend qu’un autre groupe de yoriki se sont présenté une semaine avant et qu’il les a payé.}
\clearpage
\section{Courtisans}

\subsection{Yasuki Akane}
\subsubsection*{Otage chez les pirates}

Le clan du crabe possède de nombreuses créations indestructibles. Yasuki Akane en est le parfait exemple.
Elle est une jeune samouraï de l'école Yasuki, charmante mais tenace quand il le faut. Son père est membre du conseil de Sunda Mizu Mura. 

\section{Shugenja}

\subsection{Kuni Gula}

\section{Autre}

\chapter{Le clan du Dragon}

\section{Bushi}

\subsection{Mirumoto Misara}

\section{Courtisans}

\subsection{Kitsuki Gohai}

\section{Shugenja}

\section{Autre}


\chapter{Le clan du Grue}

\section{Bushi}

\section{Courtisans}

\section{Shugenja}

\section{Autre}


\chapter{Le clan du Licorne}

\section{Bushi}

\subsection{Shinjo Toshigawua}
Voir Toritaka Toshigawua.

\subsection{Utaku Kim-li}

Utaku Kim-li est une sensei du clan de la licorne. 

\section{Courtisans}

\section{Shugenja}

\section{Autre}


\chapter{Le clan du Lion}

\section{Bushi}
\subsection{Ikoma Kae}

Le clan du lion constitue l’armée de l’empire. La richesse du clan est sa population. 
Elle lui permet de tenir quatre armées minimum. Les matsu chargent, les Akodo commandent, les Ikoma soutiennent et les Kitsu soignent et bénissent.
Voilà ce que l’empire pense des lions. \\
L’histoire du clan est parsemé d’indivue qui n’ont pas respecté les canons de leur famille. \\
Ikoma lui même fut un individu plein de faces cachées dont l’histoire n’a gardé qu’un bon côté. 
Matsu Koritome fut le premier grand archer de l’empire bien avant Tsuruchi. Il protégea l’empereur avec son talent.\\
Si l’empire a besoin d’une soldat capable de se battre en montagne le clan du lion en a. 
Si le combat se passe en mer, le lion a un général capable de commander des navires et des soldats. 
Si le combat se passe à la cour. Ils ont Ikoma Kae. Elle est l’union parfaite entre la beauté de sa mère une courtisane Ikoma et la férocité de son père Matsu.
Les closes du mariage spécifier que la première fille du couple suivrait les enseignement Matsu.
Ikoma Kae est donc une berserker du lion cependant très proche de sa mère et douée d’une grande beauté, elle fut souvent présente aux côté de sa mère dans des négociations importantes.\\
Très vite, elle fut déçue par la réputation du clan du lion et surtout de l’école Matsu. 
Beaucoup de samourais voient dans les lions des êtres qui ne cherchent que la guerre. 
Incapable de voir la beauté dans le monde environnant ou de tenir en place en période de paix.\\
Elle a mis un point d’honneur à réunir les deux aspects: la guerre et les arts.\\
Pour la guerre, elle applique les préceptes de son école. Pour les arts, elle excelle dans les arts vocaux. Le chant est son préféré mais elle ne néglige pas l’art du conteur si cher aux yeux des Ikoma.
Ikoma Kae est un défi permanent pour tout samourai qu’elle rencontre. Les bushi se laisse 
avoir par son air de courtisan. Les courtisans se laissent avoir dès qu’ils voient son Mon 
de l’école Matsu.\\
Cette dualité a attiré l’attention d’Ikoma lui même. Ce patronnage crée un désordre sentimental chez Ikoma Kae.


\section{Courtisans}

\section{Shugenja}
\subsection{Matsu Ayako}

\section{Autre}


\chapter{Le clan du Mante}

\section{Bushi}

\section{Courtisans}

\section{Shugenja}

\section{Autre}


\chapter{Le clan du Phénix}

\section{Bushi}

\subsection{Shiba Karyko}

\section{Courtisans}

\subsection{Asako Yumi}

\section{Shugenja}

\section{Autre}


\chapter{Le clan du Scorpion}

\section{Bushi}

\section{Courtisans}

\subsection{Shosuro Himoko}

\section{Shugenja}

\section{Autre}



\chapter{Le clan du Araignée}

\section{Bushi}

\subsection{Daigotsu Yukuko}
\subsection{Daigotsu Shuuki}

\section{Courtisans}

\section{Shugenja}

\subsection{Chuda Kariyami}

\section{Autre}


\chapter{Les familles Impériales}

\section{Bushi}

\subsection{Miya Akamu}

\subsection{Miya Raïku}

\section{Courtisans}

\section{Shugenja}

\section{Autre}


\chapter{Clan mineur}

\section{Bushi}

\section{Courtisans}

\section{Shugenja}

\subsection{Kitsune Hideko}
\subsubsection*{L’enfant de la forêt}

La forêt de Kitsune Mori est l’une des plus mystérieuses de tout l’empire. 
Le clan du Renard dépense beaucoup d’effort pour protéger ses secrets. 
Kitsune Hideko est un mystère issue de cette forêt. En effet, née de parents inconnus, elle a été découverte par une famille d’hemins au abord de la forêt. Elle n’était alors qu’un bébé. L’élément le plus étrange encore dans cette découverte, était la présence d’un wakizashi aux côtés de l’enfant.

La présence d’une arme aussi noble proche de l’enfant. La famille demanda à l’ancien du village de parler aux samourais. Cela prit plusieurs années avant qu’une shugenja se rende au village pour tester l’enfant.

Il fut rapidement visible pour les autorités du clan du Renard, que l’enfant avait un talent profond pour prier les kami et avec les animaux. L’enfant fut introduit au dojo. Une fille née dans la forêt ayant vécue 4 ans avec des heimins et montrant des talents supérieurs au reste de sa promotion, s’attire forcement des jalousies. 

Ne connaissant pas ses parents, elle resta attachée à sa famille adoptive et lui rendit visite dès que possible. De son enfance, elle garde une grande incompréhension de l’ordre céleste. Elle n’y est pas opposée mais elle ne le comprend pas. Elle prend avec sérieux son devoir de samourai. 

Personne ne connaît ses parents, il a été trouvé bébé par des paysans au abord de Kitsune Mori.
Il a été découvert avec a coté de lui un bout de bois (bambou) d'environ 40 cm.

Très vite, il démontre un certain talent pour communiquer avec les animaux. La nouvelle de sa découverte fait venir des samouraïs dans sa famille d'accueil. Le bout de bambou est en faite un wakizashi ayant appartenu a un ancien samouraï du clan de Renard disparue dans la foret, il y a quelques années.

Son appartenance à la classe des samourais est reconnu. Il rentre au dojo du clan. Mais pour lui ses parents restent les heimins qu'ils l'ont trouvé.

Après la vrai histoire:
-Peut-être que son pere vit en ermite dans la foret qu'il a eu un enfant avec une femme heimin et il a voulu lui permettre de devenir samourai. Il s'est débrouillé pour.
-Ou alors c'est une kitsune qu'il a engrossé.
-ou autre.

mon perso n'est pas au courant, il a pas forcement envie de connaitre la vérité. 

\section{Autre}


\chapter{Autre}

\section{Bushi}

\subsection{Saïto}

\subsection{Baito}

\subsection{Ishiro}

\subsection{Akato}



\section{Courtisans}

\section{Shugenja}

\section{Autre}

\subsection{Meshiko}


\end{document}