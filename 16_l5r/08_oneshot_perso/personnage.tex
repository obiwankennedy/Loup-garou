\documentclass[oneside,12pt]{book}
\usepackage[left=2cm,top=1cm,right=3cm,nofoot]{geometry}
\geometry{a4paper}

\usepackage[utf8]{inputenc}
\usepackage[francais]{babel}
\usepackage{tabularx}
\usepackage{graphicx}
\usepackage{epstopdf} 
\usepackage[T1]{fontenc}
\usepackage{siunitx}
\usepackage[linktocpage]{hyperref}
\usepackage{multicol}


\newcommand\sort[8]{
\begin{itemize}
\item{ \textbf{Nom}: #1}
\item{ \textbf{Élément}: #2}
\item{ \textbf{Jet}: #3}
\item{ \textbf{Portée}: #4}
\item{ \textbf{Durée}: #5}
\item{ \textbf{Zone d'effet}: #6}
\item{ \textbf{Effet}: #7}
\item{ \textbf{Augmentation}: #8}
\end{itemize}
\vspace{0.5cm}
}

\title{Personnage L5r : un nouvel espoir}
\author{Renaud "ObiWan" Guezennec}
\date{}


\begin{document}

\maketitle \clearpage
\tableofcontents \clearpage

\begin{flushleft}

\chapter{Personnages}
\clearpage
\section{Isawa Kenko}
\begin{description}
\item[Nom:]{Isawa Kenko}
\item[Clan:]{Phénix}
\item[Type:]{Shugenja (Magicien)}
\item[Age:]{18 ans}
\item[Genre:]{F}
\item[Histoire:]{
Tu es une Shugenja du clan du phénix. Le meilleur clan de rokugan, le plus savant et adepte de la magie.\\
Tu as une affiné envers les kamis du feu. Tu as naturellement pris la voie de cette élément pour ta magie.\\
Tu aimes les endroit chaud et le jour. Tu maitrises un peu moins bien la magie des autres éléments mais tu restes une Isawa, donc meilleure que les shugenja des autres clans.
Cet amour du feu ne t'a pas épargné quand tu étais jeune, tu as été brulée sur une bonne partie de ton corps. Cela n'est pas visible et cela n'est plus douloureux mais tu as pris conscience que tes amis du feu, il ne fallait pas les écouter tout le temps.\\
\vspace{0.2cm}
}
\item[Avis:]{
\underline{Daidoji Hiru} : Honorable et une force de la nature. Il est défiguré (de naissance), c'est mauvais signe, cela cache quelques choses.\\
\underline{Bayushi Sahime} : Une jeune femme talentueuse et cupide, elle est scorpionne donc dangereuse.\\
\underline{Akodo Neru} : Honorable mais tu n'aimes pas être seule avec lui. \\
\underline{Mirumoto Kitsawa} : Il n'est pas assez sorti de ses montagnes. Tout le monde est gentil pour lui. Il reste un bon combattant. \\
\underline{Hida Kokujin} : Un rustre et coureur de jupon, il sait se battre. \\
\underline{Tsuruchi Tamoe} : C'est une mante, donc ton devoir est de ne  pas l'aimer. Cependant, c'est une battante et elle garde espoir. Tu apprécies ses traits de caractère. Mais il ne faut pas le montrer.\\
}
\item[rang 1]{
Pas de défiance élémentaire. Une augmentation gratuite pour tous les sorts de feu.
}
\end{description}
\vspace{0.2cm}
Défauts:
\begin{itemize}
\item Ne peut pas mentir
\item Douillet (les malus de blessures sont augmentés de 5).
\end{itemize}
\clearpage
\textbf{\large Sorts} 
\vspace{0.2cm}


\begin{multicols}{2}
\begin{footnotesize}
\sort{Tornade}{Air 1}{3G2}{Personnelle}{1 tour}{Cône de 3m de large sur 20 de long}{1G1 de dommage dans la zone et test d'opposition terre face à l'air du lanceur. Si échec les personnes se retrouve par terre pour 1 tour}{} 

\sort{Glyphe de Protection}{Terre 1}{4G3}{Touché}{1 heure}{Une cible}{Protège une cible de la magie. Les sorts visants la cible du glyphe voient leur difficulté augmenter de 5}{Durée:+30min, +1cible pour 2 aug.}
\sort{Frappe de Jade}{Terre 1}{4G3}{30m}{Un tour}{Une cible}{3G3 de dommage si elle est souillée, sinon rien}{1 cible/aug, dommage: +1g0}
\sort{Katana de feu}{Feu 1}{6G4}{Personnelle}{Une arme créée}{5mins}{Invoque un katana, 4G3 pour toucher une cible, et le shugenja ajoute son honneur aux dommages.}{Dommage :+1g0/aug, Durée +5 }
\sort{Feu intérieur}{Feu 2}{6G4}{30m}{un tour}{une cible}{boule de feu : 3G3 de dommage}{+1 cible pour 2 augmentations} 
\sort{Extinction}{Feu 1}{6G4}{Personnel}{30m de rayon}{Un tour}{Éteint les feux dans la zone. Reduit les dommages des feux par 1G1 par tour}{+6 m de rayon par aug}
\sort{Voix de la paie intérieure}{Eau 1}{3G2}{Touché}{1 cible}{Un tour}{Soigne Résultat du dés-10 points de vie}{rien}
\sort{Sensation}{tous 1}{Dépend de l'élément choisi}{Personnel}{15m de rayon}{Un tour}{Détecte des kami de l'élément}{+3m de rayon/aug}
\sort{Invocation}{tous 1}{Dépend de l'élément choisi}{10m}{ 30 \si{\cubic\deci\meter}  }{Permanent}{Invoque l'élément: du feu, de la terre, de l'eau…}{Portée: +3m, Quantité: +30 \si{\cubic\deci\meter} , Composition de la matière: eau -> thé, la terre -> sable…}
\sort{Communion}{tous 1}{Dépend de l'élément choisi}{Personnel}{1 cible}{Concentration}{Pose 2 questions à un kami}{+1 question par aug ou rend les explications du kami plus compréhensibles}
\end{footnotesize}
\end{multicols}


\clearpage

\section{Daidoji Hiru}
\begin{description}
\item[Nom:]{Daidoji Hiru}
\item[Clan:]{Grue}
\item[Type:]{Bushi}
\item[Age:]{18 ans}
\item[Genre:]{H}
\item[Histoire:]{
Depuis tout petit, les animaux ne t'aime pas. Bébé, tu as été défiguré par un faucon. Tu as donc une cicatrice proche de l'oeil gauche, malgré cela tu es devenu un fier bushi du clan de la grue. Un guerrier de fer Daidoji. Tu as été donné comme otage dans une famille scorpionne. Tu as reçu la permission de revenir sur tes terres. Terre que tu ne connais pas bien. Bayushi Sahime est une amie d'enfance, elle est la fille de ainée du domaine où tu es otage. Elle est la pour te surveiller. En réalité, vous vous entendez plus que bien et tu lui fais confiance. 
\vspace{0.2cm}
}
\item[Avis:]{
\underline{Isawa Kenko} : une shugenja fiere et bavarde. Elle gagnerai à se taire des fois.\\
\underline{Bayushi Sahime} : Voir BG.\\
\underline{Akodo Neru} : Honorable mais tu n'aimes pas sa tronche. Il doit être un adversaire sympa\\
\underline{Mirumoto Kitsawa} : Il est un bon combattant mais utilise des techniques originales qui ne valent rien en comparaison de la tradition. \\
\underline{Hida Kokujin} : Un rustre et coureur de jupon, il sait se battre. \\
\underline{Tsuruchi Tamoe} : Elle est douée pour l'arc, elle gagnera beaucoup de tournoi. Mais les batailles ne se gagnent pas avec des flèches.\\
}
\item[rang 1]{
Ajoute (honneur-4) points de vie pour chaque rang de blessure. Gagne +1g0 sur les attaques quand tu es en position d'attaque.
}
\end{description}
\vspace{0.2cm}
Avantages:
\begin{itemize}
\item Rétablissement Rapide
\item +1 honneur
\item Géant
\end{itemize}
Défauts:
\begin{itemize}
\item Défigurer (les gens pensent que tu portes malheur).
\item Malédiction du royaume Chikushudo (le royaume des animaux): Les animaux ne t'aiment pas.
\item Otage chez des scorpions.
\end{itemize}


\clearpage

\section{Akodo Neru}
\begin{description}
\item[Nom:]{Akodo Neru}
\item[Clan:]{Lion}
\item[Type:]{Bushi}
\item[Age:]{18 ans}
\item[Genre:]{H}
\item[Histoire:]{
Tu as vu ta sœur se faire assassiné devant toi pendant une attaque de la maison familliale (tes parents sont morts pendant cette attaque aussi).
Elle s'est sacrifiée pour sauver ta vie. Tu as fuit et tu t'es retrouvé seul dans la ville, sans pouvoir prouver qui tu étais. 
Après, quelques mois une armée lionne s'est pointée dans la ville pour la raser après l'horrible meurtre de ta famille. Pendant l'assaut de l'armée, tu as suivi un des assassins que tu avais identifié. Quand tu en as eu l'occasion, tu as pris l'arme d'un samourai lion mort, et tu l'as tué.
A la fin, tu as été amené à ta tante la général qui commandait l'armée lionne. Elle t'a reconnu et tu as retrouvé ton rang.\\
Le katana que tu portes est l'arme de ta sœur. 
\vspace{0.2cm}
}
\item[Avis:]{
\underline{Isawa Kenko} : une shugenja fiere et bavarde. Elle gagnerai à se taire des fois.\\
\underline{Bayushi Sahime} : Elle sait se battre mais son école ne vaut rien comparer à la tienne dans un dojo en tout cas. A la court, méfiance.\\
\underline{Daidoji Hiru} : Honorable mais tu n'aimes pas sa tronche. Il doit être un adversaire sympa\\
\underline{Mirumoto Kitsawa} : Il est un bon combattant mais utilise des techniques originales qui ne valent rien en comparaison de la tradition. \\
\underline{Hida Kokujin} : Il sait se battre pour ça tu le respectes. Pour le reste, il n'est pas digne d'être samourai. \\
\underline{Tsuruchi Tamoe} : Une bonne combattante, elle se fit trop à son talent. Elle oublie qu'un combattant seul ne fait pas la différence. La discipline et l'entrainement voilà ce qui gagne une bataille. \\
}
\item[rang 1]{
Le bonus de ND donné par une armure est ignoré. De plus, tu gagnes +1g0 à la premiere attaque de chaque combat ou contre tout opposant qui déclare une augmentation contre toi. 
}
\end{description}
\vspace{0.2cm}
Avantages:
\begin{itemize}
\item Lame Akodo
\item +1 honneur
\item Stratège
\end{itemize}
Défauts:
\begin{itemize}
\item Phobie : solitude
\item Oublié du bushido : Sincérité
\item Idéaliste
\end{itemize}

\clearpage

\section{Bayushi Sahime}
\begin{description}
\item[Nom:]{Bayushi Sahime}
\item[Clan:]{Scorpion}
\item[Type:]{Bushi}
\item[Age:]{18 ans}
\item[Genre:]{F}
\item[Histoire:]{
Tu es une bushi du clan du scorpion, ta maîtrise du katana n'est pas mauvaise mais tu te distingues plus facilement à la cour. Tu attires les regards, les gens ont souvent un bon «feeling» avec toi.
Ta famille est l'hôte de Daidoji Hiru. Il est arrivé chez vous quand tu étais toute petite. Vous avez le même âge. Vous vous appréciez bien l'un l'autre.
\vspace{0.2cm}
}
\item[Avis:]{
\underline{Isawa Kenko} : une shugenja fiere et bavarde. Elle gagnerai à se taire des fois.\\
\underline{Akodo Neru} : Honorable donc manipulable mais avec précaution.\\
\underline{Daidoji Hiru} : voir bg\\
\underline{Mirumoto Kitsawa} : Il est un bon combattant, il est probablement très manipulable. \\
\underline{Hida Kokujin} : Un rustre et coureur de jupon, il sait se battre. Une cible trop facile. \\
\underline{Tsuruchi Tamoe} : Une rivale intéressante, pas encore à la hauteur mais sa voix est à une emprise importante. \\
}
\item[rang 1]{
Gagne +1g1 à l'initiative. Si tu as l'initiative par rapport à ton adversaire ton nd est augmenté de +5.
}
\end{description}
\vspace{0.2cm}
Avantages:
\begin{itemize}
\item Benten : +1g1 en social
\item Chanceux : peut relancer un jet/séance
\item Silencieux : +1g0 en jet de discrétion. 
\end{itemize}
Défauts:
\begin{itemize}
\item Cupide:+1g1 à l'adversaire pour un jet de tentation (corruption).
\item Présomptieux: ND 20 en perception pour ne pas courir dans la bataille.
\item Petit: -1g0 aux dommages
\end{itemize}


\clearpage

\section{Mirumoto Kitsawa}
\begin{description}
\item[Nom:]{Mirumoto Kitsawa}
\item[Clan:]{Dragon}
\item[Type:]{Bushi}
\item[Age:]{18 ans}
\item[Genre:]{H}
\item[Histoire:]{
Tu vises à devenir un dueliste hors du commun. Pour l'instant, tu n'es pas mauvais mais tu es loin d'être remarquable. Par contre, en dehors du duel, tu maîtrises tes 2 lames comme il faut.\\
Tu es un peu perdu face à toutes ces nouveautés. Tu avais l'habitude de tes montagnes. Les gens des basses terres ne sont pas si simple à comprendre que ceux de ton clan. Cela te manque mais tu sais que c'est ici que tu pourras servir le mieux l'empire. 
\vspace{0.2cm}
}
\item[Avis:]{
\underline{Isawa Kenko} : une shugenja fière et bavarde. Les kami du feu sont très puissante en elle.\\
\underline{Akodo Neru} : Honorable mais il ne cherche pas l'illumination. Un bon combattant mais tu dois être capable de le battre en seul à seul.\\
\underline{Daidoji Hiru} : Honorable, il ne t'inspire pas confiance.\\
\underline{Bayushi Sahime} : Elle est gentille, et agréable. Tu ne mettras pas en doute sa capacité de combattante mais tu es meilleur. \\
\underline{Hida Kokujin} : Un rustre et coureur de jupon, il sait se battre. \\
\underline{Tsuruchi Tamoe} : Le meilleur archer que tu es jamais vu. Elle nourrit de grand espoir. C'est fatiguant. \\
}
\item[rang 1]{
Quand tu utilises tes deux armes tu gagnes un bonus au ND équivalent au double de ton rang et tu ne subit aucun malus dans l'usage des deux armes. Plus, tu peux réduire ou augmenter la difficulté des sorts de 5 (pour les sorts qui te visent)
}
\end{description}
\vspace{0.2cm}
Défauts:
\begin{itemize}
\item Constitution de la terre : moins de malus
\item Rétablissement rapide
\item Toucher de Tengoku +2g0 pour résister à la souillure.
\end{itemize}
Défauts:
\begin{itemize}
\item Ascete
\item Idéaliste
\item Naif
\item Peur des navires (1)
\end{itemize}

\clearpage

\section{Hida Kokujin}
\begin{description}
\item[Nom:]{Hida Kokujin}
\item[Clan:]{Crabe}
\item[Type:]{Bushi}
\item[Age:]{18 ans}
\item[Genre:]{H}
\item[Histoire:]{
Tu descends d'une grande famille de combattant du mur. Tu rêves d'y accomplir ton devoir et d'y mourir devant un adversaire à ta taille. Cependant, ta mission est tout autre pour le moment. Tu es avec des samourais qui dénigrent la tache de ton clan, et qui pensent que les crabes ne sont pas fait pour s'intégrer dans le reste de l'empire. Tu aimerais prouver le contraire. Cependant, tu aimes faire la fête et profiter des plaisirs de la chair. Du coup, c'est compliqué.
\vspace{0.2cm}
}
\item[Avis:]{
\underline{Isawa Kenko} : une shugenja fière et bavarde. Tu n'existes pas pour elle.\\
\underline{Akodo Neru} : Bon combattant, il te respecte pour ta valeur de combattant mais rien d'autre.\\
\underline{Daidoji Hiru} : Honorable, il ne t'inspire pas confiance.\\
\underline{Bayushi Sahime} : Tu te méfis, elle est charmante. Tu pourrais succomber facilement à ses charmes.\\
\underline{Mirumoto Kitsawa} : Il semble perdu en dehors de ses montagnes. Il se bat de façon bizarre. \\
\underline{Tsuruchi Tamoe} : Femme douée, elle ira loin. Elle sait utiliser un arc mais cela ne passe pas une armure lourde. \\
}
\item[rang 1]{
Tu n'as aucun malus pour porter une armure lourde sauf en discrétion. Quand tu utilises une arme lourde, tu gagnes +1g0. 
}
\end{description}
\vspace{0.2cm}
Défauts:
\begin{itemize}
\item Virtuose
\item Poing de Pierre
\end{itemize}
Défauts:
\begin{itemize}
\item Peur Poney/Chevaux
\item Jolie Cœur
\end{itemize}


\clearpage

\section{Tsuruchi Tamoe}
\begin{description}
\item[Nom:]{Tsuruchi Tamoe}
\item[Clan:]{Mante}
\item[Type:]{Bushi}
\item[Age:]{18 ans}
\item[Genre:]{F} 
\item[Histoire:]{
Tu es une archère Tsuruchi. Tu es samourais mais tu ne portes pas de daisho (katana + wakizashi). Du coup, pour les samourais sont assez médisant à ton propos.
Tu souhaites leur prouver le contraire. Tu es très douée à l'arc. Tu espères te faire un nom grâce à ton arc. Tu connais par coeur l'histoire de Matsu Koritome. Il sauva la vie de l'empereur grâce à son arc en tuant tous les gaijins. Tu aimerais bien racontre des archers lions de cette famille vassale. Tsuruchi avait une mère Lionne. La légende raconte qu'elle était une descendante de Matsu Koritome.
\vspace{0.2cm}
}
\item[Avis:]{
\underline{Isawa Kenko} : une shugenja fière et bavarde. Tu n'existes pas pour elle.\\
\underline{Akodo Neru} : Bon combattant, il te respecte pour ta valeur de combattant mais rien d'autre.\\
\underline{Daidoji Hiru} : Honorable, il ne t'inspire pas confiance.\\
\underline{Bayushi Sahime} : Tu te méfis, elle est charmante et toi aussi. Elle te voit comme une rivale.\\
\underline{Mirumoto Kitsawa} : Il semble perdu en dehors de ses montagnes. Il se bat de façon bizarre comme toi mais pas pareil. \\
\underline{Hida Kokujin} : Un rustre et coureur de jupon, il sait se battre. \\
}
\item[rang 1]{
+3 à l'initiative et +1g0 pour les attaques à l'arc. 
}
\end{description}
\vspace{0.2cm}
Défauts:
\begin{itemize}
\item Jeune Prodige (kyu)
\item Benten
\item Voix harmonieuse
\item Vedette
\end{itemize}
Défauts:
\begin{itemize}
\item Insensible
\item Compulsion (les tournois d'archers)
\item Obnubilé (Devenir la meilleure archère de l'empire)
\end{itemize}


\end{flushleft}


\end{document}
