% !TEX encoding = UTF-8 Unicode

%!TEX TS-program = xelatex
%!TEX encoding = UTF-8 Unicode

\documentclass[oneside,12pt]{book}
\usepackage[left=2cm,top=1cm,right=3cm,nofoot]{geometry}                % See geometry.pdf to learn the layout options. There are lots.
\geometry{a4paper}                   % ... or a4paper or a5paper or ... 
\usepackage{tabularx}

\usepackage{fontspec,xltxtra,xunicode}
\defaultfontfeatures{Mapping=tex-text}
\usepackage[french]{babel}
\usepackage{listings}
\usepackage{graphicx}
\usepackage[linktocpage]{hyperref}



\title{Le roi est mort! vive les rois}
\author{Renaud "ObiWan" Guezennec}
\date{}

%%%%%%%%%%%%%%%%%%%%%
%
% Les Lieux
%
%%%%%%%%%%%%%%%%%%%%%


\newcommand{\Guyenne}{\textbf{Guyenne} }%territoire du royaume de france, sous controle anglais.
\newcommand{\France}{\textbf{France} }
\newcommand{\Navarre}{\textbf{Navarre} }
\newcommand{\Europe}{\textbf{Europe} }
\newcommand{\Angleterre}{\textbf{Angleterre} }
\newcommand{\Provence}{\textbf{Provence} }
\newcommand{\Bretagne}{\textbf{Bretagne} }
\newcommand{\Normandie}{\textbf{Normandie} }
\newcommand{\Ecosse}{\textbf{Écosse} }
\newcommand{\Flandre}{\textbf{Flandre} }
\newcommand{\AbbayeSD}{\textbf{Abbaye de Saint-Denis} }
\newcommand{\SaintSardos}{\textbf{Saint-Sardos en Guyenne} }
\newcommand{\Calais}{\textbf{Calais} }
\newcommand{\Reims}{\textbf{Reims} }
%%%%%%%%%%%%%%%%%%%%%
%
% Les francais
%
%%%%%%%%%%%%%%%%%%%%%

\newcommand{\PhilipeIVLeBel}{\textbf{Philipe IV le bel} }%père de Louis X
\newcommand{\JeanneI}{\textbf{Jeanne 1ère de Navarre} }%épouse de Philipe le bel

%fils de philipe le bel.
\newcommand{\LouisX}{\textbf{Louis X le Hutin} }%roi mort au début, 1er fils de philipe IV
\newcommand{\MargueriteB}{\textbf{Marguerite De Bourgogne} }%épouse de Louis X
\newcommand{\Clemence}{\textbf{Clemence de Hongrie} }%épouse de Louis X

\newcommand{\PhilipeV}{\textbf{Philipe V} }% 2eme fils de philipe IV
\newcommand{\CharlesIV}{\textbf{Charles IV} }% 3eme fils de philipe IV
\newcommand{\IsabelleDeFrance}{\textbf{Isabelle De France} }% 1ere fille de philipe IV, epouse de Edouard II d'angletrre, née juste après Louis X

%Enfant de Louis X.
\newcommand{\Jeanne}{\textbf{Jeanne Reine de Navarre} }%Fille de louis X
\newcommand{\JeanI}{\textbf{Jean Ier} }%fils de louis X, mort après 4 jours






%%%%%%%%%%%%%%
\newcommand{\CharlesDeValois}{\textbf{Charles De Valois} }%frere de Philipe IV le bel
\newcommand{\PhilipeVI}{\textbf{Philipe VI} }% fils de Charles De valois


%Enfant de PhilipeVI
\newcommand{\JeanII}{\textbf{Jean II le bon} }
\newcommand{\MarieDeFrance}{\textbf{Marie de France} }
\newcommand{\LouisDeFrance}{\textbf{Louis de France} }
\newcommand{\LouisBDeFrance}{\textbf{Louis de France} }
\newcommand{\JeanDeFrance}{\textbf{Jean de France} }
\newcommand{\PhillipeDeFrance}{\textbf{Philippe de France} }
\newcommand{\JeanneDeFrance}{\textbf{Jeanne de France} }


%%%%%%%%%%%%%%%%%%%%%
%
% Bretagne
%
%%%%%%%%%%%%%%%%%%%%%
\newcommand{\JeanDeMonfort}{\textbf{Jean de Montfort} }%  
\newcommand{\CharlesDeBlois}{\textbf{Charles de Blois} }




%%%%%%%%%%%%%%%%%%%%%
%
% Les anglais
%
%%%%%%%%%%%%%%%%%%%%%

\newcommand{\EdouardII}{\textbf{Edouard II} }%roi anglais à la mort de LouisX
\newcommand{\EdouardIII}{\textbf{Edouard III d'Angleterre} }%roi anglais à la mort de LouisX
\newcommand{\EdouardWoodstock}{\textbf{Édouard de Woodstock} }

%%%%%%%%%%%%%%%%%%%%%
%
% Navarre 
%
%%%%%%%%%%%%%%%%%%%%%

\newcommand{\CharlesLeMauvais}{\textbf{Charles Le Mauvais Roi de Navarre} }%fils de \Jeanne
\newcommand{\PhillipeDeNavarre}{\textbf{Phillipe De Navarre} }%2ème fils de \Jeanne

%%%%%%%%%%%%%%%%%%%%%
%
% PNJ 
%
%%%%%%%%%%%%%%%%%%%%%

\newcommand{\ducGuyenne}{\textbf{Raymond-Bernard de Montpezat} }
\newcommand{\RoiEcosse}{\textbf{David roi d'Ecosse} }

\newcommand{\DeLaCerda}{\textbf{Charles De La Cerda} }

\newcommand{\Peste}{\textbf{La peste noire} }
\newcommand{\Normand}{\textbf{La peste noire} }


\newcommand{\Anglais}{\textbf{anglais\\} }

\begin{document}

\maketitle \clearpage
\tableofcontents \clearpage

\begin{flushleft}
    \chapter{Introduction}
        \section{Introduction sur l'univers}

		\section{Situation economique des pays concernés}


\chapter{L'histoire commence}

\section{Mort du roi}
Le roi \LouisX roi de \France Meurt, il ne laisse derrière lui qu'une fille. Sa femme est enceinte, il faut la protéger. 
La meilleure idée pour la protéger, c'est de la faire rentrer en \Provence, elle y sera en sécurité.

\Jeanne est une fille, et trop jeune pour régner,
 la condanation de sa mère \MargueriteB n'est qu'un prétexte pour l'écarter du pouvoir. 
On préfère attendre l'accouchement, de \Clemence et la naissance de l'enfant. 
C'est \PhilipeV frère de \LouisX qui assurera la régence.



\subsection{Séance 1 : Protection de la reine}
Ici, les joueurs auront un scénario assez actions, ils auront pour mission de livrer \Clemence, dans un couvent en \Provence.
Ils auront quelques problèmes dans le trajet. 


\section{Le petit roi est mort, vive la reine… ou pas…, vive le frère du roi}
\Jeanne est définitivement écartée du pouvoir car le régent \PhilipeV devient roi de  \France.
C'est choix géopolitique, on refuse de voir un étranger épouser \Jeanne et devenir roi de \France.
Le règne de \PhilipeV dure peu et il ne laisse pas d'héritier. 

\subsection{Séance 2 : Les textes de lois}
Ici, les joueurs auront pour taches de trouver un héritier en accord avec les lois. Il est fort possible que la lois utile n'existe pas. 
Il faudra la créer. Ils seront envoyer dans un monastère/bibliothèque/archive nationale pour trouver un texte de loi. 
S'il ne le trouve pas, il faudra le fabriquer.


\section{Le roi est mort, vive le frère du roi (bis)}
\CharlesIV devient roi de \France.
\CharlesIV aura trois épouse dont sa cousine germaine…
Son règne dure très peu de temps. 

\subsection{Séance 3 : prise d'une bastide en \Guyenne}
Les pj accompagne une expédition de \CharlesDeValois, en \Guyenne. \CharlesDeValois prend un fort à \SaintSardos.
Le seigneur local \ducGuyenne, vient porter réclamation est demande un jugement. 
Alors que les PJs repartent avec le gros des troupes, 
\ducGuyenne attaque la bastide et passe tout le monde au fils de l'épée et le représentant de \CharlesIV est pendu.


\section{Le roi est mort, vive le cousin}
La seule descendence par les males est assurée par le cousin \PhilipeVI (il a 2 fils).
\EdouardIII revendique le pouvoir du royaume de \France au titre qu'il est le seul petit fils de \PhilipeIVLeBel. 
Avec réticence, il prête hommage à \PhilipeVI pour garantir le maintien de ses terres \Guyenne sur le royaume \France. 
En échange, il espère en échange récupérer les terre d' \Ecosse. 
\PhilipeVI refuse de céder ces terres, \EdouardIII prends se pretexte pour déclarer la guerre. 


\section{\RoiEcosse envahi les îles anglo-normades}
    Pour faire peur au roi \EdouardIII, \RoiEcosse attaque des petits territoires annexes au royaume d'\Angleterre. C'est un échec militaire mais un gros avertissement à \EdouardIII. 
 
 \subsection{Le jour J}
     Les pj, pourrait participer à l'attaque qui a un tout autre objectif en réalité que celui connu dans les livres d'histoire. 
     Voir récupérer un object précieux au yeux du roi. Tuer sa maitresse. 
      

\section{Jeu d'alliance}
    \EdouardIII s'allie avec la \Flandre, qui ont des accords très mal équilibrée avec la \France. 
Les artisant de \Flandre s'enfuit en \Angleterre et crée une industrie la bas.
 La \Flandre déclare \EdouardIII roi de \France.  


    \subsection{C'est de la contrebande ?}
    Les joueurs pourrait être amener à enquêter sur ces transactions. 
Fouiller des bateaux et comprendre l'interet de \EdouardIII à tous cela. 
Cela peut être aussi la découverte d'une lettre de gage affrimant la reconnaissance de \EdouardIII comme roi de \France. 
    
\section{Declaration de guerre}
 Suite à l'attaque sur les Îles et maintenant que l'économie va mieux en \Angleterre,
 \EdouardIII déclare la guerre à la \PhilipeVI. Il se proclame roi de France.
Un archeveque vient jeter le gant à la figure de celui qui "se proclame roi de \France". 

\subsection{Tuer le messager}
Le roi de \France, ordonne que l'on tue l'archèque de manière discrète. Avant qu'il ne raconte à toute l'\Europe ce qu'il a fait. 

\section{jeu d'alliance bis}
\EdouardIII prends le partie de soutenir \JeanDeMonfort . Le roi de \France soutient \CharlesDeBlois adversaire direct de 
\JeanDeMonfort pour la succession du duché de \Bretagne.

\section{La flotte arrive}
\France et des mercenaires forment une grande flotte et essaie de bloquer les principaux ports de l'\Angleterre et de ses comptoirs. 

\subsection{Pillage dans un port \Angleterre}
Les Pj pourrait être amener à débarquer en \Angleterre. Pour tisser des liens avec des opposants du roi. 

\section{Défaite de la bataille des Ecluses}
La flotte de \France est largement détruite. Le commerce avec l'\Angleterre reprend. 

\subsection{Courage fuyons vite}
Les PJs vont peut-être assisté à la défaite et ils devront s'en sortir. Il sera très compliqué aux PJ de retourner chez eux. 
Rubrique survival. Ils entendront surement que les \Anglais ont débarqué proche de leur route. 

\section{Les victoires de \EdouardIII}
Dans les premieres années du conflit, le génie militaire d'\EdouardIII ridiculise la noblesse de \France. Les batailles rangées 
sont clairement favorable à l'armée d'\Angleterre. La \France doit agir autrement. 

\subsection{C'est la merde}
Les pj qui seront probablement envoyés en campagne vont entendre parler des attaques. 
Les pillages font rage. Il auront certainement à aider un peu la population. 
Ce sera probablement sur les terres d'un des joueurs. Il voudra sauver ses paysans.
Il est possible de faire un scénario, ou les PJ doivent aider les civils à passer une fleuve ou une rivière, pour se mettre à 
l'abri des pillage. Les \Anglais détruisent le bétail, les outils de production 

\section{Echec aux rois }
La strastégie du roi de \France: Soit garnir les chateaux et autres places fortes, cette méthode coute chere. Les pillages des 
chevauchées et mercenaires \Anglais plus la solde des soldats en garnison est très onéreuse. L'autre stratégie est de former la 
population pour chasser les \Anglais. Les premières batailles rangés démontre la désorganisation des armées du royaume de 
\France. La noblesse veut faire des rançons pour se refaire une santé financière. Ils n'obléissent pas au roi.  
La tentative de stoper les chevauchées échoue. 

\subsection{Le dementellement}
Les PJ pourrait être témoins d'une bataille, les patrons des pj pourrait mourir dans la bataille ou leur fils.

\section{\EdouardIII et son armée}
Apres avoir anéanti l'armée du royaume de \France, \EdouardIII fait le siège de \Calais.
\Calais ouvre ses portes quand ils apprenent que le roi de \France ne pourra venir les sauver car son armée de secours est bien 
trop petite pour faire lever le siège.
 
\subsection{Délivrer le message à \Calais}
Les pj iront porter un message au dirigeant de la ville de \Calais. Soit au nom de quelqu'un qui n'a pas envie de voir la ville 
rasée car il y a des intérêts, soit par le roi de \France.  

\section{trêve: un peu de repos}
\EdouardIII rentre en \Angleterre. \Calais est à lui. 

\section{La \Peste frappe et les rats quittent le navire}
La \France est frappée par une épidémie de \Peste. Cette maladie est pour beaucoup une punition divine. La noblesse s'est faite 
décapité et n'a plus les moyens d'assurer la sécurité des routes et des domaines. Les soldats mercenaires sont démobilisés par 
les \Anglais, ils forment de grandes compagnies qui volent et sacage les villes et villages de \France. 
Les pairs de \France vote d'une voie le maintien de \PhilipeVI sur le trône. Le roi \PhilipeVI meurt très rapidement son fils 
\JeanII prend la couronne. 

\subsection{D'ou vient la \Peste}
Les joueurs peuvent enquété sur la nature de la \Peste. Il pourrait apprendre que c'est un démon, ou les \Anglais qui l'ont 
apporté ou juste la faute à pas de chance. 

\section{Magouille pendant la trève}
\DeLaCerda mène une flotte qui repousse (malgrés de lourde pertes) la flotte du royaume \Angleterre et  \EdouardIII qui voulait se faire 
couronnée à \Reims.
\DeLaCerda devient un ami proche du roi \JeanII (voir très proche). 
\JeanII ne veut pas rompre la trève et engage des pour parler avec \EdouardIII, pour une paix durable. \DeLaCerda sera chargé de 
négocier la paix pour le roi (ou d'établir le premier contact).  
\CharlesLeMauvais envoie son frère \PhillipeDeNavarre capturer \DeLaCerda. Cependant l'attaque de l'auberge ou réside \DeLaCerda 
se passe dans la confusion est  \PhillipeDeNavarre ordonne la mort de \DeLaCerda. \CharlesLeMauvais revendique l'assassinat par 
chevalerie. Il recueille ainsi le soutien de ses voisins \Normand. En effet, le roi de \Navarre a des terres en \Normandie.
A l'annonce de la mort de \DeLaCerda, \JeanII restre cloitré 5 jours dans son chateau. 
\JeanII est obligé d'accepter les revendications territoriale de \CharlesLeMauvais.
 










 

\section{idée en vrac}
A la fin le nouveau fait une loi pour autoriser une femme à regner sur le territoire (si univers l'autorise)







\end{flushleft}
\end{document}
