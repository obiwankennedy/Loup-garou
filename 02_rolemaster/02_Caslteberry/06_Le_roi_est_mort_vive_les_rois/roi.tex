% !TEX encoding = UTF-8 Unicode

%!TEX TS-program = xelatex
%!TEX encoding = UTF-8 Unicode

\documentclass[oneside,12pt]{book}
\usepackage[left=2cm,top=1cm,right=3cm,nofoot]{geometry}                % See geometry.pdf to learn the layout options. There are lots.
\geometry{a4paper}                   % ... or a4paper or a5paper or ... 
\usepackage{tabularx}

\usepackage{fontspec,xltxtra,xunicode}
\defaultfontfeatures{Mapping=tex-text}
\usepackage[french]{babel}
\usepackage{listings}
\usepackage{graphicx}
\usepackage[linktocpage]{hyperref}



\title{Le roi est mort! vive les rois}
\author{Renaud "ObiWan" Guezennec}
\date{}

%%%%%%%%%%%%%%%%%%%%%
%
% Les Lieux
%
%%%%%%%%%%%%%%%%%%%%%


\newcommand{\Guyenne}{\textbf{Guyenne} }%territoire du royaume de france, sous controle anglais.
\newcommand{\France}{\textbf{France} }
\newcommand{\Navarre}{\textbf{Navarre} }
\newcommand{\Europe}{\textbf{Europe} }
\newcommand{\Angleterre}{\textbf{Angleterre} }
\newcommand{\Provence}{\textbf{Provence} }
\newcommand{\Ecosse}{\textbf{Écosse} }
\newcommand{\Flandre}{\textbf{Flandre} }
\newcommand{\AbbayeSD}{\textbf{Abbaye de Saint-Denis} }
\newcommand{\SaintSardos}{\textbf{Saint-Sardos en Guyenne} }
%%%%%%%%%%%%%%%%%%%%%
%
% Les francais
%
%%%%%%%%%%%%%%%%%%%%%

\newcommand{\PhilipeIVLeBel}{\textbf{Philipe IV le bel} }%père de Louis X
\newcommand{\JeanneI}{\textbf{Jeanne 1ère de Navarre} }%épouse de Philipe le bel

%fils de philipe le bel.
\newcommand{\LouisX}{\textbf{Louis X le Hutin} }%roi mort au début, 1er fils de philipe IV
\newcommand{\MargueriteB}{\textbf{Marguerite De Bourgogne} }%épouse de Louis X
\newcommand{\Clemence}{\textbf{Clemence de Hongrie} }%épouse de Louis X

\newcommand{\PhilipeV}{\textbf{Philipe V} }% 2eme fils de philipe IV
\newcommand{\CharlesIV}{\textbf{Charles IV} }% 3eme fils de philipe IV
\newcommand{\IsabelleDeFrance}{\textbf{Isabelle De France} }% 1ere fille de philipe IV, epouse de Edouard II d'angletrre, née juste après Louis X

%Enfant de Louis X.
\newcommand{\Jeanne}{\textbf{Jeanne Reine de Navarre} }%Fille de louis X
\newcommand{\JeanI}{\textbf{Jean Ier} }%fils de louis X, mort après 4 jours


%%%%%%%%%%%%%%
\newcommand{\CharlesDeValois}{\textbf{Charles De Valois} }%frere de Philipe IV le bel
\newcommand{\PhilipeVI}{\textbf{Philipe VI} }% fils de Charles De valois



%%%%%%%%%%%%%%%%%%%%%
%
% Les anglais
%
%%%%%%%%%%%%%%%%%%%%%

\newcommand{\EdouardII}{\textbf{Edouard II} }%roi anglais à la mort de LouisX
\newcommand{\EdouardIII}{\textbf{Edouard III d'Angleterre} }%roi anglais à la mort de LouisX




%%%%%%%%%%%%%%%%%%%%%
%
% PNJ 
%
%%%%%%%%%%%%%%%%%%%%%

\newcommand{\ducGuyenne}{\textbf{Raymond-Bernard de Montpezat} }
\newcommand{\RoiEcosse}{\textbf{David roi d'Ecosse} }

\begin{document}

\maketitle \clearpage
\tableofcontents \clearpage

\begin{flushleft}
    \chapter{Introduction}
        \section{Introduction sur l'univers}

		\section{Situation economique des pays concernés}


\chapter{L'histoire commence}

\section{Mort du roi}
Le roi \LouisX roi de \France Meurt, il ne laisse derrière lui qu'une fille. Sa femme est enceinte, il faut la protéger. 
La meilleure idée pour la protéger, c'est de la faire rentrer en \Provence, elle y sera en sécurité.

\Jeanne est une fille, et trop jeune pour régner, la condanation de sa mère \MargueriteB n'est qu'un prétexte pour l'écarter du pouvoir. 
On préfère attendre l'accouchement, de \Clemence et la naissance de l'enfant. C'est \PhilipeV frère de \LouisX qui assurera la régence.



\subsection{Séance 1 : Protection de la reine}
Ici, les joueurs auront un scénario assez actions, ils auront pour mission de livrer \Clemence, dans un couvent en \Provence.
Ils auront quelques problèmes dans le trajet. 


\section{Le petit roi est mort, vive la reine… ou pas…, vive le frère du roi}
\Jeanne est définitivement écartée du pouvoir car le régent \PhilipeV devient roi de  \France.
C'est choix géopolitique, on refuse de voir un étranger épouser \Jeanne et devenir roi de \France.
Le règne de \PhilipeV dure peu et il ne laisse pas

\subsection{Séance 2 : Les textes de lois}
Ici, les joueurs auront pour taches de trouver un héritier en accord avec les lois. Il est fort possible que la lois utile n'existe pas. Il faudra la créer. 


\section{Le roi est mort, vive le frère du roi (bis)}
\CharlesIV devient roi de \France.
\CharlesIV aura trois épouse dont sa cousine germaine…
Son règne dure très peu de temps. 

\subsection{Séance 3 : prise d'une bastide en \Guyenne}
Les pj accompagne une expédition de \CharlesDeValois, en \Guyenne. \CharlesDeValois prend un fort à \SaintSardos.
Le seigneur local \ducGuyenne, vient porter réclamation est demande un jugement. 
Alors que les PJs repartent avec le gros des troupes, \ducGuyenne attaque la bastide et passe tout le monde au fils de l'épée et le représentant de \CharlesIV est pendu.


\section{Le roi est mort, vive le cousin}
La seule descendence par les males est assurée par le cousin \PhilipeVI (il a 2 fils).
\EdouardIII revendique le pouvoir du royaume de \France au titre qu'il est le seul petit fils de \PhilipeIVLeBel. Avec réticence, il prête hommage à \PhilipeVI pour garantir le maintien de ses terres \Guyenne sur le royaume \France. En échange, il espère en échange récupérer les terre d' \Ecosse. 
\PhilipeVI refuse de céder ces terres, \EdouardIII prends se pretexte pour déclarer la guerre. 


\section{\RoiEcosse envahi les îles anglo-normades}
    Pour faire peur au roi \EdouardIII, \RoiEcosse attaque des petits territoires annexes au royaume d'\angleterre. C'est un échec militaire mais un gros avertissement à \EdouardIII. 
 
 \subsection{Le jour J}
     Les pj, pourrait participer à l'attaque qui a un tout autre objectif en réalité que celui connu dans les livres d'histoire. 
     Voir récupérer un object précieux au yeux du roi. 
      

\section{Jeu d'alliance}
    \EdouardIII s'allie avec la \Flandre, qui ont des accords très mal équilibrée avec la \France. Les artisant de \Flandre s'enfuit en \Angleterre et crée une industrie la bas. La \Flandre déclare \EdouardIII roi de \France.  

    \subsection{C'est de la contrebande ?}
    Les joueurs pourrait être amener à enquêter sur ces transactions. Fouiller des bateaux et comprendre l'interet de \EdouardIII à tous cela. Cela peut être aussi la découverte d'une lettre de gage affrimant la reconnaissance de \EdouardIII comme roi de \France. 
    
\section{Declaration de guerre}
 Suite à l'attaque sur les Îles et maintenant que l'économie va mieux en \Angleterre, \EdouardIII déclare la guerre à la \PhilipeIV. Il se proclame roi de France.
Un archeveque vient jeter le gant à la figure de celui qui "se proclame roi de \France". 

\subsection{Tuer le messager}
Le roi de \France, ordonne que l'on tue l'archèque de manière discrète. Avant qu'il ne raconte à toute l'\Europe ce qu'il a fait. 




\section{idée en vrac}
A la fin le nouveau fait une loi pour autoriser une femme à regner sur le territoire (si univers l'autorise)







\end{flushleft}
\end{document}
